\documentclass{article}
\usepackage[a4paper, margin=1in]{geometry}

\newcommand{\topic}[1]{\large{\textbf{#1}}\normalsize}

\title{Database Systems Lecture 1}
\date{August 24, 2021}
\author{Nicolas Andrew Burnett}

\begin{document}
  \maketitle
  \section{Syllabus}
  \begin{itemize}
    \item HW(4) 10\%, Projects(2) 25\%, SQL Assignments(2) 10\%, Attendance 5\%, Midterm 20\%, Final 30\%
    \item Exams will be 40\% FRQ, 60\% MCQ
    \item Exams at testing center. Midterm \textbf{Oct 8th}
    \item Attendance is mandatory
  \end{itemize}
  \section{Chapter 1}
  \topic{What is a database?}
  \begin{itemize}
    \item Means of persisting data \& non-volatile memory. Persists between RAM cycles. Stored in disk space rather than RAM
    \item Has information about entities, their attributes, and their relationship to other entities
    \item Relational databases allow the maintanence of relationships between database entities
    \item Problem domain may be called \textbf{Business Domain or Universe of Discourse}
  \end{itemize}
  \topic{What is a Database Management System?}
  \begin{itemize}
    \item DBMS is a server or application that acts as an interface to the database
    \item Provides services used to \textbf{CRUD} entity instances
    \item DBMS' provide services to one or more cients. Can maintain more than more database
    \item Provides languages and tools used to define the structure of entities (schema) and interact with them
    \item Protects databases from data corruption using built-in rules or through some sort of interface-level sanitization
    \item \textbf{Data Definition Language (DDL)} is used to describe entities
    \item Entities are defined by attributes, and entities maintain associations with other entities
  \end{itemize}
  \topic{MATTHEW IS RIGHT THIS MAN MAKES ANY TOPIC THE MOST BORING, BOILERPLATE TOPIC IN THE UNIVERSE}
\end{document}