\documentclass{article}
\usepackage[a4paper, margin=1in]{geometry}
\usepackage{listings}
\usepackage{color}

\lstset{frame=tb}
\newcommand{\topic}[1]{\large\textbf{#1}\normalsize}
\newcommand{\java}[0]{\lstset{language=Java}}
\newcommand{\python}[0]{\lstset{language=Python}}

\title{Programming Language Paradigms Lecture 3}
\date{August 30, 2021}
\author{Nicolas Andrew Burnett}

\begin{document}
  \maketitle
  \section{Chapter 1 (cont)}
  \topic{Influence of Language Design}
  \begin{itemize}
    \item Major influence is computer architecture. von Neumann influenced imperative languages, used today
    \item Programmming design methodologies also influence language design\begin{itemize}
      \item top-down design. lead to work on type checking and structured programming
      \item data-oriented design methodologies. lead to abstract data-types, object-oriented programming
      \item concurrent programming. shifted focus back to procedural-oriented design
    \end{itemize}
  \end{itemize}
  \topic{Language Categories}
  \begin{itemize}
    \item imperative (java, c++, python)
    \item functional (scheme, haskal)
    \item logic
    \item object-oriented
  \end{itemize}
  \topic{Implementation Methods}
  \topic{Compilation}\begin{itemize}
    \item starts with the source
    \item sent to lexical analysis (detects keywords, create groups of characters)
    \item sent to syntax analysis (analyze groups and ensure order is correct)
    \item sent to intermediate code (turned into low-level code, but not machine code. may also optimize)
    \item generates machine code (takes intermediate code and converts to machine code, generates executable)
  \end{itemize}
  \topic{Pure Interpretation}
  \begin{itemize}
    \item Runs exactly what it's given. Executes while parsing.
  \end{itemize}
  \topic{Hybrid}
  \begin{itemize}
    \item starts from source
    \item lexical analysis
    \item sytax analysis
    \item intermediate code (considered 'virtual machine code')
    \item intermediate code is fed into an interpretor.
  \end{itemize}
  \section{Chapter 3}
  Chapter 3 will formally define the formation and semantics of a program.
  \newline
  \topic{Syntax}
  \begin{itemize}
    \item \textbf{Alphabet:} a set of symbols
    \item \textbf{String:} a sequence of symbols from some alphabet
    \item \textbf{Language:} a set of strings made from some alphabet
    \item \textbf{Sentence:} a string in a language
    \item \textbf{Lexeme:} description of lowest-level syntactic unit
    \item \textbf{Token:} a category of lexemes (dentifiers, keywords, operators, etc)
    \begin{itemize}
      \item two ways to formally define languages
      \item \textbf{Recognition:} check if a string is in the language
      \item \textbf{Generation:} generate a sentence of a language
    \end{itemize}
    \item \textbf{Grammars:} known as generators, tells how to generate a string

    \topic{Backus-Naur Form}
    \begin{itemize}
      \item context-free grammar, simply sees circumstances and decides what a symbol is
      \item \textbf{Terminal Symbol:} a symbol from the alphabet, lexeme, or token
      \item \textbf{Nonterminal Symbol:} a symbol that can be substituted via a production rule
    \end{itemize}
    \topic{Describing Lists}
    \begin{itemize}
      \item usually utilizes a form of recursion
    \end{itemize}
    \topic{Derivations}
    \begin{itemize}
      \item start with start symbol
      \item apply rule to substitute a Nonterminal
      \item repeat until only terminals are left
      \item to derive a pecific string, choose the right rules
      \item \textbf{Sentential Form:} strings in the derivation
      \item \textbf{Leftmost Derivation:} always choose the left most nonterminal to substitute
      \item \textbf{Rightmost Derivation:} always choose the right most nonterminal to substitute
    \end{itemize}
    \topic{Parse Trees}
    \begin{itemize}
      \item lose order of substitute information (no leftmost or rightmost)
      \item gain hierarchical structure information
      \item any informational order is lost
    \end{itemize}
  \end{itemize}
\end{document}