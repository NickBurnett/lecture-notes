\documentclass{article}
\usepackage[a4paper, margin=1in]{geometry}
\usepackage{listings}
\usepackage{color}

\lstset{frame=tb}
\newcommand{\java}[0]{\lstset{language=Java}}
\newcommand{\python}[0]{\lstset{language=Python}}

\title{Programming Language Paradigms Lecture 1}
\date{August 23, 2021}
\author{Nicolas Andrew Burnett}

\begin{document}
  \maketitle
  \section{Syllabus}
  \java
  \begin{lstlisting}
  // Main.java
  public static void main(String[] args) {
    System.out.println(15);
  }
  \end{lstlisting}
  \begin{itemize}
    \item 3 Exams 27\% (multiple choice, 30 questions), HW 32\%, Projects 32\%, Quizzes 9\%
    \item Programming will happen, but it's disecting the theoretical aspects of programming language \textbf{design and implementation}
    \item Book is \textbf{dense and hard}, meant for reference, not to learn from
    \item Biggest difficulty is \textbf{Chapter 3}. Lots of material, heavy math element. \textbf{Chapter 15/16} is \textit{heaviest}, cover different programming paradigms (\textbf{imperative} programming, \textbf{functional} programming, \textbf{logical} programming)
    \item Projects look intimidated, but is meant for exploration. Will be an interpreter
    \item Assignments will be turned in through eLearning (\textbf{PDF/TXT format}, anything else won't be graded. Programs are turned in using \textbf{Native Language Files} (.py, .java, .js, .ts, etc))
    \item Grade boosts are available to those who attend class
  \end{itemize}
\end{document}